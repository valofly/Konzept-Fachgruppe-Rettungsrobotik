\section{Aufgaben}

Die \callee{} ist zur Unterstützung von anfordernden Stellen konzipiert. Der Fokus liegt in der Erkundung von Einsatzlagen sowie der Aufbereitung und Darstellung der gewonnenen Informationen, möglichst in Echtzeit, für die Einsatzführung.\\

\noindent Die \callee{} kann zur Erkundung der Einsatzlage (S2) sowie dem Abarbeiten von Aufgaben mit Einsatzbezug (S3) eingesetzt werden. Darüber hinaus kann die \callee{} die Kreispressegruppe im Rahmen der Presse- und Medienarbeit (S5) unterstützen, wobei erstelltes Bild- oder Videomaterial ausschließlich durch die Kreispressegruppe verarbeitet und herausgegeben werden darf.\\

\noindent Die \callee{} kann die Koordination weiterer an der Einsatzstelle agierender \ac{uav} von gleichgestellten, anerkannten und\,/\,oder mitwirkenden Hilfsorganisationen sowie Dritten übernehmen.

\subsection{Leistungsspektrum für die Einsatzleitung}

Die \callee{} unterstützt die Einsatzleitung maßgeblich in der Erkundung sowie der Einsatzdokumentation. In sehr begrenztem Maße ist die \callee{} befähigt auch aktiv in den Einsatz einzugreifen.

\subsubsection{Erkundung}

Die \callee{} hilft bei der Erkundung der Einsatzstelle. Dies umfasst z. B. Glutnester in schwer zugänglichen Bereichen von Brandobjekten, Gefahrstoffdetektion bei Objekten von denen besondere Gefahren ausgehen oder Strahlenquellen.\\

\noindent Mit Bildgebenden verfahren können Lagekarten der Einsatzleitung um Realbilder der Einsatzstelle erweitert werden, um z. B. versperrte oder zerstörte Wege hinsichtlich des Zustandes zu bewerten.

\subsubsection{Dokumentation}

Die \callee{} kann für die Einsatzleitung Teile des Einsatzablaufes, das Schadensbild an der Einsatzstelle oder die Aufstellung der Einheiten vor Ort als Luftbildfotografie dokumentieren.

\subsubsection{Funkanwendungen}

Im Falle von großflächigen Schadenslagen kann das \ac{tetra} Netz überlastet oder ausgefallen sein. Über die \ac{uav} in Verbindung mit Tethersystemen lassen sich Relaisstationen für \ac{dmo} realisieren oder zerstörte \ac{tetra} Basisstationen bis zur Bereitstellung eines Ersatzneubaus ersetzen, um eine funktionierende Infrastruktur zur Kommunikationsübermittlung aufrechtzuerhalten.
