\section{Ausrüstung}

Das Konzept beschreibt hinsichtlich der Ausrüstung den Soll-Zustand der \callee{}. Dieser ist regelmäßig mit dem Ist-Zustand der \callee{} abzugleichen. Die Differenz ist durch den Gruppenführer und seinen Stellvertreter zu bewerten und im Rahmen von Bedarfsanalysen für eine Beschaffung nach Dringlichkeit zu bewerten.

\subsection{Persönliche Schutzausrüstung}

Die Gruppe ist in der einheitlichen Schutzausrüstung der \unit{} einzukleiden.
Aufgrund einer reduzierten Gefährdung kann bei Übungsdiensten und im Einsatzfall vom Tragen der vollständigen Schutzausrüstung zur Trageerleichterung abgesehen werden. Eine Gefährdungsbeurteilung ist separat zu erstellen.
Abweichend von der persönlichen Schutzausrüstung gelten für den Fernpiloten sowie den Einsatzraum- / Luftraumbeobachter folgende Besonderheiten:

\begin{itemize}
\item Auf einen Helm kann zugunsten einer Cappy oder Bergmütze verzichtet werden. Dies begründet sich in der Notwendigkeit, in den Himmel blicken zu können. Eine entsprechende Kopfbedeckung verhindert hierbei Einschränkungen der Sicht durch Blendungen. Das Helmgewicht selbst stellt durch das anhaltende in die Luft schauen ein Verletzungsrisiko dar.
\item Tragen von Sonnenbrillen ohne Polarisationsfilter um das Risiko von Blendungen weiter zu reduzieren. Die Lesbarkeit von Bildschirmen muss gewährleistet sein.
\item Zur Steuerung von unbemannten Fahrzeugen kann auf das Tragen von Handschuhen verzichtet werden, um eine sichere und genaue Bedienung zu gewährleisten.
\end{itemize}

\subsection{Geräte und Material}

Die \callee{} ist mit den folgend genannten Systemen auszustatten. Die einzelnen Punkte werden jeweils kurz begründet.\\

\noindent Die Beschaffungen sind so zu gestalten, dass die Systeme unter gewöhnlichen Umständen möglichst lange nutzbar bleiben. Es muss jedoch darauf hingewiesen werden, dass nahezu alle beschriebenen Systeme starken technologischen Entwicklungen unterliegen und bereits nach wenigen Jahren veraltet sein können. Die Gruppe ist angehalten, Komponenten möglichst plattform- und herstellerunabhängig in den Beschaffungsplanungen zu berücksichtigen. Beispielhaft sollte eine \ac{wbk} auch am Nachfolgemodell eines \ac{uav} einsetzbar sein.

\subsubsection{Unbemannte Systeme}

Die \callee{} ist zu befähigen, die Einsatzleitung mit unbemannten Systemen in der Luft, zu Land und unter Wasser zu unterstützen. Hierfür sind die folgend genannten Systeme erforderlich.

\paragraph{UAV — Multi-Role}

Die \callee{} ist mit einem Multi-Role \ac{uav}\footnote{\url{https://de.wikipedia.org/wiki/Multicopter}} auszustatten. Der Fokus des Systems liegt auf einem hohen \ac{mtow} und einer möglichst modularen Schnittstelle zur Ansteuerung von Sensorik, um ein möglichst breites Spektrum an Aufgaben abzudecken. Das UAV sollte ebenfalls für den kabelgebundenen Flug (tethering) vorbereitet sein, um einen durchgängigen Betrieb und damit eine unterbrechungsfreie Lagenbildübermittlung zu gewährleisten.

\paragraph{UAV — Singe-Purpose}

Die \callee{} ist mit mindestens einem günstigeren \ac{uav} auszustatten, für welches das Risiko einer Ersatz-Neubeschaffung bereits einkalkuliert ist. Dieses \ac{uav} soll im Rahmen der Rettung von Menschenleben als verhältnismäßige Option zur Verfügung stehen, auch wenn mit dem Verlust des Systems gerechnet werden muss. Das Luftfahrzeuge sollten über mindestens eine RGB-Kamera sowie eine Infrarotkamera verfügen.

\paragraph{UGV — Multi-Role}

Für Einsätze auf ebener Fläche wird ein \ac{ugv}\footnote{\url{https://de.wikipedia.org/wiki/Unbemanntes_Landfahrzeug}} benötigt. Mit diesem können z. B. in \acs{cbrn}-Lagen ohne explosive Atmosphären Messgeräte für temporäre oder dauerhafte Messungen in einen Gefahrenbereich verbracht werden. Daten die durch bildgebende Verfahren gewonnen werden, können der Einsatzleitung dauerhaft bereitgestellt werden. Mit Greifarmen und entsprechenden Manipulatoren ausgestattet können einfache Hindernisse beseitigt werden, um eine Erkundung durchzuführen.
Mit den gewonnenen Informationen kann der Personaleinsatz in der Erkundung reduziert und die einzusetzenden Trupps folglich besser auf Ihren Einsatz vorbereitet werden.

\paragraph{UUV — Multi-Role}

Für Einsätze unter Wasser wird ein \ac{uuv}\footnote{\url{https://de.wikipedia.org/wiki/Ferngesteuertes_Unterwasserfahrzeug}} benötigt, welches folgende Aufgaben abdecken kann: bildgebende Verfahren, Arbeitsscheinwerfer, Greifwerkzeuge für Transport von Werkzeug, Sonartechnik. Das System soll im Bedarfsfall zur Unterstützung der Taucher eingesetzt werden. Ein Einsatz kann jedoch ausschließlich unter Aufsicht durch geschultes Personal erfolgen.

\subsubsection{Kommunikationstechnik}
\paragraph{Funkgeräte}

Im Landkreis ist für den Betrieb von unbemannten Fahrzeugen für Ausbildung und Einsätze die Rufgruppe \dmoGroup{} (\acs{dmo}) vorgesehen.
Die Notwendigkeit dieser Maßnahme besteht in der möglichen Dringlichkeit der Nachrichtenübermittlung im Falle von plötzlich oder unvorhergesehen auftretenden Gefahrenquellen.

\paragraph{Netzwerktechnik}

Für die Herstellung der Kommunikation mit Führungsfahrzeugen sollte die \callee{} mit technischer Infrastruktur ausgestattet werden, die das Erstellen eines lokalen Einsatznetzes mit hoher Bandbreite ermöglicht. Der Up-/Downstream über LTE ist dabei als Fallback zu betrachten, da dieser im Falle einer Katastrophenlage ausgefallen oder überlastet sein kann.

\subsubsection{Allgemeine Ausrüstung}

\paragraph{Start- und Landeplatz}

Es sind zwei Sätze zur Markierung von Start- und Landeplätzen vorzuhalten. Diese enthalten jeweils vier selbstleuchtende Markierungen zum Absichern des Start- und Landebereich sowie benötigtes Zubehör.

\paragraph{Stromerzeuger}

Auf dem Fahrzeug ist ein Stromerzeuger mit 4\,kW bis\,6 kW Leistung vorzuhalten. Strom wird für das Fahrzeug inklusive der Computer- und Netzwerktechnik, für das Laden von Akkuzellen sowie zur Versorgung einer Tether Bodenstation benötigt.

\paragraph{Batteriespeicher}

Zusätzlich zum Stromerzeuger ist eine batteriegepufferte Wechselstromerzeugung als Redundanz vorzuhalten. Die Option zum Laden über ein Solar-Panel ist empfehlenswert. Müssen Trupps unabhängig vom Fahrzeug arbeiten, ist so das Laden von Akkuzellen gewährleistet.

\subsection{Software}

Es fallen unter Umständen Lizenzgebühren für einzelne Softwarepakete an. Für gängige Software sollte auf Open-Source-Software zurückgegriffen werden.\\

\noindent Die Beschaffung und Verwendung von spezieller Software ist dem Standard der Kreisfeuerwehr \district{} anzupassen.

\subsection{Fahrzeug}

Das Fahrzeug der \callee{} ist mindestens geländefähig und bietet Platz für mindestens vier Personen. Das Fahrzeug verfügt über zwei Computerarbeitsplätze mit je zwei Bildschirmen sowie ausreichend Stauraum für Technik.\\

\noindent Das Fahrzeug benötigt Breitband-Mobilfunk mit einer Mindestübertragungsgeschwindigkeit von 150 Mbit/s (4G, besser 5G Standard), um Informationen über das Internet teilen und empfangen zu können.\\

\noindent Das Fahrzeug ist mit zwei \ac{mrt} sowie separaten Antennen für 1090\,MHz, 2,4\,GHz und 5\,GHz auszustatten. An den Computerarbeitsplätzen sollte ein \ac{mrt} für TMO sowie \ac{mrt} für \ac{dmo} vorgehalten sein. Die Antenne mit 1090\,MHz dient der vom Internet unabhängigen Erfassung von Flugbwegungen über \acs{adsb}\footnote{\url{https://de.wikipedia.org/wiki/Automatic_Dependent_Surveillance}}. Über die Antennen im 2,4\,GHz und 5\,GHz Frequenzband lassen sich unbemannte Systeme vom Fahrzeug aus steuern oder ein WLAN zur Datenübertragung im Nahbereich des Fahrzeugs aufspannen.\\

\noindent Das Fahrzeug kann den landschaftlichen Gegebenheiten entsprechend im gesamten Landkreis \district{} eingesetzt werden. Mit dem Fahrzeug lassen sich alle Einsätze gemäß Kapitel \ref{sec:alarmSmall} abdecken. Dabei ist das Fahrzeug dafür ausgelegt, auch in umfassenderen Einsatz- und Katastrophenlagen die Nutzung von unbemannten Systemen über die eigenen Möglichkeiten der \callee{} hinaus zu koordinieren.
