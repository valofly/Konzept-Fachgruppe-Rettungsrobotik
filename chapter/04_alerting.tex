\section{Alarmierung}

Die Alarmierung der \callee{} erfolgt über \ac{dme} sowie, sofern individuell akzeptiert, per App über die persönlichen Smartphones.\\

\noindent Verfügt ein Mitglied über keinen \ac{dme} (z. B. über die Ortsfeuerwehr), so ist dieser über den Landkreis in angemessener Zeit bereitzustellen.

\subsection{Alarmstichworte}

Die \callee{} kann über die Alarmstichworte \flq Einsatz Fachberater\frq{}, \flq Robotik klein\frq{} oder \flq Robotik groß\frq{} alarmiert werden. Das Alarmstichwort \flq Akkus Laden\frq{} dient lediglich organisatorischen Zwecken.\\

\noindent Über das Alarmstichwort wird primär festgelegt, wie viele Einsatzmittel, also unbemannte Systeme, parallel einsetzbar sind.

\subsubsection{Einsatz Fachberater}

Der Fachberater kann zur Beratung oder Klärung, ob ein Einsatz der \callee{} möglich ist, alarmiert werden. Ferner können vergleichbare Einheiten auf Stadtebene unterstützt werden.

\paragraph{Einsatzmittel} Es kann kein Einsatzmittel eingesetzt werden.

\paragraph{Kräfteeinsatz} Der Fachberater rückt (\,0\,/\,1\,/\,0\,/\,\underline{1}\,) aus.

\paragraph{Alarmschleife} Die Alarmschleife ist auf dem \ac{dme} des Gruppenführers und seinem Stellvertreter hinterlegt.

\subsubsection{Robotik klein}
\label{sec:alarmSmall}

\noindent Der Kräfteeinsatz umfasst die Führung sowie einen \robotiktrupp{}. Zusätzlich erweitert der \melder{} in diesem Fall als Truppführer den \robotiktrupp{} zu einem selbstständigen Trupp. Der \melder{} ist für die Dokumentation und das Akkumanagement verantwortlich.

\paragraph{Einsatzmittel} Es kann ein Einsatzmittel eingesetzt werden.

\paragraph{Kräfteeinsatz} Die \callee{} rückt (\,0\,/\,1\,/\,3\,/\,\underline{4}\,) aus. Weitere Kräfte verbleiben in Bereitstellung an Wache.

\paragraph{Alarmschleife} Die Alarmschleife ist auf den \ac{dme} aller Mitglieder der \callee{} hinterlegt.

\subsubsection{Robotik groß}

Der Kräfteeinsatz umfasst den \gruppenfuehrer{}, \melder{}, \maschinist{}, \angriffstrupp{} und \wassertrupp{} sowie den \schlauchtrupp{}.

\paragraph{Einsatzmittel} Es können zwei Einsatzmittel parallel eingesetzt werden.

\paragraph{Kräfteeinsatz} Die \callee{} rückt (\,0\,/\,1\,/\,8\,/\,\underline{9}\,) aus. Die \callee{} kann Ausrücken sobald (\,0\,/\,1\,/\,3\,/\,\underline{4}\,) an Wache sind und Eile geboten ist. Weitere Kräfte müssen mit einem MTW an die Einsatzstelle verbracht werden. Weitere Kräfte verbleiben in Bereitstellung an Wache.

\paragraph{Alarmschleife} Die Alarmschleife ist auf den \ac{dme} aller Mitglieder der \callee{} hinterlegt.

\subsubsection{Akkus Laden}

Die Alarmschleife \flq Akkus Laden\frq{} startet den Ladevorgang der Akkus, an mit dem Melder verbundenen Ladegeräten. Über das Auslösen dieses Alarmstichwortes kann mit dem Laden der Akkus vor Dienstbeginn begonnen werden, sodass es im Ausbildungsbetrieb keine Verzögerung durch nicht oder unvollständig geladene Akkus gibt.

\paragraph{Einsatzmittel} Es kann kein Einsatzmittel eingesetzt werden.

\paragraph{Kräfteeinsatz} Es werden keine Einsatzkräfte alarmiert.

\paragraph{Alarmschleife} Die Alarmschleife ist auf \ac{dme}, die Ladegeräte steuern, hinterlegt.

\subsection{Alarm- und Ausrückeordnung}

Die Alarmierung der \callee{} erfolgt entsprechend der \ac{aao}.\\

\noindent Die nachfolgende Übersicht ist eine grobe Empfehlung bei welchen Stichworten die \callee{} automatisch und mit welcher Alarmierung zu Einsätzen hinzugezogen werden kann. Die Hinterlegung der Alarmstichworte ist unabhängig vom vorliegenden Konzept und wird deshalb nicht mit dem jeweiligen Status Quo in diesem Dokument berücksicht.\\

\begin{table}[h]
\begin{tabular}{lll}
\textbf{Art} & \textbf{Name} & \textbf{Stichwort} \\ \hline
	Freiflächenbrand
	& Waldbrand
	& Robotik groß
\\
\rowcolor[HTML]{EFEFEF}
	Sondereinsatzmittel
	& SoKFZ TEL Brand
	& Robotik klein
\\
	Sondereinsatzmittel
	& SoKFZ TEL TH
	& Robotik klein
\\
\rowcolor[HTML]{EFEFEF}
	Bahneinsatz
	& Zugunglück
	& Robotik groß
\\
	Gefahrstoffeinsatz
	& Gefahrstoffeinsatz groß
	& Robotik klein
\\
\rowcolor[HTML]{EFEFEF}
	sonstige Einsätze
	& Personensuche
	& Einsatz groß
\\
\end{tabular}
\end{table}

\vspace*{1\baselineskip}

\noindent Ein Einsatz bei größeren Bränden, bei Einsätzen der Fachgruppe SRHT oder bei Taucheinsätzen kann durch die Leitstelle empfohlen werden.
