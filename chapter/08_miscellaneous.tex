\section{Sonstiges}

\subsection{Standort}

Die \callee{} ist an der \ac{ftz} Einbeck stationiert.

\subsection{Mitgliedschaft}
\label{sec:membership}

Die Beantragung der Mitgliedschaft erfolgt durch Ausfüllen eines Mitgliedsformulars. Über die Aufnahme in die \callee{} entscheiden die Gruppenführer. Eine Ablehnung der Mitgliedschaft ist gegenüber antragstellenden Personen durch die Gruppenführer zu begründen.\\

\noindent Mitglied der \callee{} kann werden, wer

\begin{itemize}
\item das 18. Lebensjahr vollendet hat,
\item aktives Mitglied einer Feuerwehr ist,
\item die Qualifikationen gem. Kapitel \ref{sec:qualifikationen} erfüllt und
\item im Landkreis \district{} oder nahe der Kreisgrenze wohnhaft ist.
\end{itemize}

\noindent Die Mitgliedschaft der \callee{} endet ohne Frist durch Austritt des Mitglieds in schriftlicher Form oder Tod.\\

\noindent Zum Jahresende ist durch die Gruppenführer die prozentuale Anwesenheit bei regulären Ausbildungsdiensten festzustellen. Mitglieder, bei denen diese unter 55\,\% liegt, werden durch die Gruppenführer um Stellungnahme gebeten. Tritt innerhalb von sechs Monaten nach der Bitte um eine Stellungnahme keine Besserung der Dienstbeteiligung ein und liegt keine Begründung zum Fernbleiben von Ausbildungsdiensten vor, die dieses rechtfertigt, wird die Mitgliedschaft in der \callee{} beendet.

\subsection{Betreiber-ID für UAV}

Die Betreiber-ID für unbemannte Systeme, zugeteilt durch das \ac{lba} an\\

\indent\operatorName{}\\
\indent\operatorStreet{}\\
\indent\operatorZIP{} \operatorCity{}\\

\noindent lautet: \textbf{\operatorID{}}. Diese ist in Textform oder als QR-Code an allen \ac{uav} der \callee{} anzubringen.

\subsection{Wappen und Patches}

Den Mitgliedern der \callee{} ist es gestattet, auf der Dienstkleidung das Wappen der \flq Feuerwehr Landkreis \unit{}\frq{} zu tragen.\\

\noindent Den Mitgliedern ist es gestattet, auf Einsatzkleidung ein Patch der \callee{} zu tragen. Andere Kennzeichen, Wappen und Patches, als die vom Gruppenführer freigegebenen, sind auf allen Teilen der Dienst- und Einsatzkleidung untersagt.

\subsection{Abrechnung und Kosten}
\label{sec:accounting}

Den Mitgliedern der \callee{} steht je Ausbildungsdienst eine Erstattung der Fahrtkosten sowie eine Verpflegungspauschale zu. Die jeweilige Höhe sowie die Abrechnungsintervalle sind durch den Landkreis \district{} festgelegt. Es besteht kein Anspruch aus nicht dokumentierten Teilnahmen an Ausbildungsdiensten.\\

\noindent Es können keine Ansprüche aus Fahrten zu Einsätzen geltend gemacht werden.

\subsection{Versicherungsschutz}

Für Mitglieder der \callee{} gilt der übliche Versicherungsschutz bei Ausbildungsdiensten und Einsätzen über die \ac{fuk}.\\

\noindent Ein separater Versicherungsschutz für Haftpflichtschäden aus Unfällen mit unbemannten Systemen ist über den \ac{ksa} abgedeckt.\\

\noindent Es obliegt dem Landkreis \district{} die Ausrüstung der \callee{}, gegebenenfalls über die gesetzlich vorgeschriebenen Versicherungen hinausgehend, zu versichern.

\subsection{Open Data}

Von Mitgliedern der \callee{} erstellte Dokumente können unter \textit{CC\,BY-NC-SA}\footnote{\url{https://creativecommons.org/licenses/by-nc-sa/4.0/}} oder \textit{CC\,0}\footnote{\url{https://creativecommons.org/publicdomain/zero/1.0/}} veröffentlicht werden. Die Lizenz kann jeweils frei gewählt werden.\\

\noindent Die Mitglieder der \callee{} dürfen nur solche Informationen unter CC-Lizenzen veröffentlichen, welche nicht durch Dokumente des Landkreises \district{}, durch anders lautende Lizenzen oder Gesetze und Verordnungen geschützt sind. Wappen und Logos sind zu entfernen.\\

\noindent Dieses Dokument ist in einer Musterfassung unter \textit{CC\,0} veröffentlicht\footnote{\url{https://github.com/valofly/Konzept-Fachgruppe-Rettungsrobotik}}. Die Veröffentlichung soll dem Austausch und der Diskussion zum Einsatz von unbemannten Systemen durch Behörden und Organisationen mit Sicherheitsaufgaben dienen sowie für Neugründungen vergleichbarer Einheiten den Start erleichtern.
