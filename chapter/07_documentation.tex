\section{Dokumentation}

Sowohl im Ausbildungsbetrieb als auch im Einsatz sind Vorgänge und Abläufe zu dokumentieren. Die Dokumentationen die im Ausbildungsbetrieb vorgenommen werden sind auch im Einsatz zu erfüllen.\\

\noindent Das vorliegende Konzept geht nur auf die Dokumentationen ein die über die üblichen Dokumentationen, wie z. B. das Führen von Fahrtenbüchern, hinausgehen.

\subsection{Dokumentation im Ausbildungsbetrieb}

Jeder Fernpilot hat ein Flugbuch zu führen, dass nach aktueller Rechtsprechung des Luftrechtes folgende Inhalte vorweisen muss: Name des Fernpiloten, verwendetes UAV, Datum, Zeit von Start und Landung, Information über besondere Vorkommnisse. Die Aufbewahrungsfrist eines Eintrages beträgt mindestens fünf Jahre und ist den zuständigen Behörden auf Verlangen vorzulegen. Es gilt nach NfL 2-330-17 als Nachweis erbrachter Flugleistungen.\\

\noindent Im Ausbildungsbetrieb sowie in Einsätzen werden Flüge mit unbemannten Systemen analog zu Atemschutzprotokollen geführt. Die Fernpiloten erhalten regelmäßig eine Kopie der aufgezeichneten Daten.\\

\noindent Es ist zur Ermittlung der prozentualen Anwesenheit (Kapitel \ref{sec:membership}) sowie zur Abrechnung der Kosten (Kapitel \ref{sec:accounting}) erforderlich, bei Ausbildungsdiensten eine Anwesenheitsliste zu führen. Diese ist quartalsweise beim Zugführer einzureichen.

\subsection{Dokumentation im Einsatz}

Neben dem Einsatztagebuch sind folgende Dokumentationen zu führen und dem Einsatztagebuch beizufügen:

\begin{itemize}
\item Flugprotokoll
\item Sicherheitschecklisten zu jedem Start
\item Einsatzauftrag zu jedem Start
\item Funkprotokoll der gegebenen Befehle
\end{itemize}
