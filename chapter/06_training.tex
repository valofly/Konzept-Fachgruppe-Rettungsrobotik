\section{Ausbildung}

Die \callee{} trifft sich im zweiwöchigen Rhythmus. Sonderdienste können im angemessenen Rahmen ohne Genehmigung der Kreisfeuerwehr unter voller Anrechnung der regulären Kosten eines Dienstes durchgeführt werden.

\subsection{Schwerpunkte der Ausbildung}

Die Ausbildung der \callee{} sollte jährlich die folgende Schwerpunkte in angemessener Gewichtung von theoretischen und praktischen Inhalten abdecken.

\subsubsection{Rechtskunde}

Die gültige Rechtsgrundlage ist jährlich auszubilden. Der Ausbildungsdienst umfasst sowohl die Kenntnis der notwendigen Gesetze sowie der für die \callee{} wichtigen Paragrafen und Absätze sowie die Auswirkung dieser auf die Arbeit im Ausbildungsdienst sowie im Einsatz.\\

\subsubsection{Einsatzgrundlagen}

Die Einsatzgrundlagen umfassen alle notwendigen Tätigkeiten im Einsatzumfeld. Dazu gehören die Fahrzeugaufstellung, die Wahl von geeigneten Start- und Landeplätzen für unbemannte Systemen, die jeweils benötigte Dokumentation von Tätigkeiten sowie die Inhalte des vorliegenden Konzeptes.

\subsubsection{Kommunikations- und Netzwerktechnik}

Im Rahmen der Kommunikationstechnik sind die physikalischen Grundlagen von Funktechnik auszubilden. Ebenso soll rudimentäres Wissen über Netzwerktechnik und den Aufbau von IP-Netzwerken vermittelt werden. Der Fokus der Ausbildung liegt auf der Bedienung der notwendigen Softwaresysteme sowie dem übertragen und bereitstellen der gewonnen Informationen an andere Einheiten.

\subsubsection{Datenauswertung}

Damit die Daten durch Dritte richtig interpretiert werden können, müssen diese entsprechend aufbereitet werden.\\

\noindent Es werden Kenntnisse über die vorhandene Sensorik vermittelt, wie diese funktionieren, welche Messbereiche möglich sind und unter welchen Bedingungen mit welchen Fehlertoleranzen gerechnet werden muss.

\subsubsection{Wetterkunde}

Das Wetter hat einen wichtigen Einfluss auf unbemannte Systeme. Im Rahmen der Wetterkunde werden die Einsatzgrenzen der vorhandenen Systeme ausgebildet. Ausbildungsinhalt ist das Bewerten des Wetters und lokal durch die Einsatzstelle verursachte Wetterphänomene auf die verwendeten unbemannten Systeme.

\subsubsection{Robotik UAV}

Die Schulung zur Nutzung von unbemannten Flugsystemen ist in theoretische und praktische Einheiten aufzuteilen.\\

\noindent Der theoretische Teil der Ausbildung umfasst die Grundlagen zur Steuerung der Systeme, den technischen Aufbau der Basiskomponenten sowie die jeweiligen Einsatzgrenzen der vorhandenen Systeme.\\

\noindent Der praktische Teil unterteilt sich in Trainingseinheiten die eine sichere Handhabung der Systeme ermöglichen. Hierzu zählen einfache Flugmuster oder Flüge um Hindernisse. Darauf aufbauend werden manuell geflogene Einsatzszenarien trainiert. Der letzte Teil umfasst das Planen und Durchführen von autonomen Flügen.

\subsubsection{Robotik UGV}

Die Schulung zur Nutzung von unbemannten Bodensystem ist in theoretische und praktische Einheiten aufzuteilen.\\

\noindent Der theoretische Teil der Ausbildung umfasst die Grundlagen zur Steuerung der Systeme, den technischen Aufbau der Basiskomponenten sowie die jeweiligen Einsatzgrenzen der vorhandenen Systeme.\\

\noindent Der praktische Teil unterteilt sich in Trainingseinheiten die eine sichere Handhabung der Systeme ermöglichen. Hierzu zählen Bewegungen um Hindernisse und einfache Übungen zur sicheren Handhabung angeschlossener Systemkomponenten. Darauf aufbauend werden komplexere Einsatzszenarien trainiert.

\subsubsection{Robotik UUV}

Die Schulung zur Nutzung von unbemannten Unterwassersystemen ist in theoretische und praktische Einheiten aufzuteilen.\\

\noindent Der theoretische Teil der Ausbildung umfasst die Grundlagen zur Steuerung der Systeme, den technischen Aufbau der Basiskomponenten sowie die jeweiligen Einsatzgrenzen der vorhandenen Systeme.\\

\noindent Der praktische Teil unterteilt sich in Trainingseinheiten die eine sichere Handhabung der Systeme ermöglichen.

\subsection{Ausbildung mit anderen Einheiten}

Die \callee{} sollte halbjährlich mindestens einen Übungsdienst gemeinsam mit den Zügen der \unit{} absolvieren.\\

\noindent Die \callee{} sollte jährlich mindestens einen Übungsdienst
oder eine Alarmübung mit anderen Einheiten des Landkreises \district{} absolvieren.\\

\noindent Die \callee{} sollte mit den \ac{tel} des Landkreises \district{} mindestens einen gemeinsamen Ausbildungsdienst durchführen und eine Alarmübung eingebunden werden.
