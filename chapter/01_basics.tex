\section{Grundlagen}
\subsection{Allgemeines}

Für den Einsatz und die Ausbildung zum Führen von unbemannten Fahrzeugen in der \callee{} des Landkreises \district{} gelten die vom Bundesamt für Bevölkerungsschutz veröffentlichten \flq Empfehlungen für gemeinsame Regelungen zum Einsatz von Drohnen im Bevölkerungsschutz\frq{}, der DIN 5452-1 \flq Luft- und Raumfahrt – Unbemannte Luftfahrzeugsysteme  (UAS) - Teil 1: Begriffe\frq{}\footnote{\url{https://www.beuth.de/de/norm/din-5452-1/325927409}}, der DIN 5452-2 \flq Luft- und Raumfahrt – Unbemannte Luftfahrzeugsysteme (UAS) - Teil 2: Anforderungen an Fernpiloten\frq{}\footnote{\url{https://www.beuth.de/de/norm/din-5452-2/311210255}} sowie alle weiteren dafür notwendigen gesetzlichen Bestimmungen der Bundesrepublik Deutschland.\\

\noindent Die Entwicklung der rechtlichen Situation im Bereich der unbemannten Wasser- und Landfahrzeuge ist genauestens zu verfolgen, da viele Regulierungen auf Landes- und Bundesebene sowie innerhalb der Europäischen Union zurzeit allenfalls als Entwurf vorliegen. Sollten diese in Kraft treten, ist das vorliegende Konzept einer Prüfung zu unterziehen und an die dann geltenden Vorgaben anzupassen.\\

\noindent Derzeit liegen keine \acp{fwdv}, welche den Einsatz von unbemannten Systemen im Bereich der Feuerwehren regulieren, vor. Für den Einsatz sind zum aktuellen Zeitpunkt jedoch insbesondere die FwDV 100 \flq Führung und Leitung im Einsatz\frq{} und FwDV 800 \flq Informations- und Kommunikationstechnik im Einsatz\frq{} zu beachten.

\subsection{Verbindlichkeit}

Das Einsatzkonzept ist für alle Feuerwehren im Landkreis \district{} und ihre Mitglieder bindend. Im Falle der Zusammenarbeit mit anerkannten und/oder mitwirkenden Hilfsorganisationen sowie Regieeinheiten oder Feuerwehren außerhalb des Landkreises \district{} ist zu prüfen, welche Einheit den Einsatz führt und auf Basis welcher Grundlagen gearbeitet wird.\\

\noindent Sollte dieses Konzept nicht zum Tragen kommen, kann die \specialistGroup{} nach einer Einweisung in die geltenden Bestimmungen eingesetzt werden, insofern die rechtlichen Grundlagen berücksichtigt werden.

\subsection{Rechtliche Grundlagen}

Es gilt das Recht der Bundesrepublik Deutschland, sowie die Gesetze des Landes \state{}. Hinsichtlich der Arbeit im Rahmen des Konzeptes sind dabei insbesondere das \acf{bdsg}, die \acf{dsgvo} sowie im Rahmen des Möglichen auch das \acf{natsg} zu Berücksichtigen.

\subsubsection{Luftfahrzeuge}

Im Bereich der unbemannten Luftfahrzeuge sind zu berücksichtigen:
\begin{itemize}
\item \acf{luftvg}
\item \acf{luftvo}
\item \acf{luftvzo}
\item Verordnung zur Regulierung des Betriebes von unbemannten Fluggeräten
\item \acf{sera}
\item \acf{notam}\footnote{\url{https://de.wikipedia.org/wiki/NOTAM}}
\item \acf{nfl}\footnote{\url{https://de.wikipedia.org/wiki/Nachrichten_f\%C3\%BCr_Luftfahrer}}
\item \acf{aic}\footnote{\url{https://de.wikipedia.org/wiki/Flugberatungsdienst}}
\end{itemize}

\noindent Hervorzuheben ist der Abschnitt 5a der \ac{luftvo}\footnote{\url{https://www.gesetze-im-internet.de/luftvo_2015/}}, insbesondere §\,21k\,\ac{luftvo}, in der Sonderregelungen bei Not- und Unglückslagen geregelt werden. Die dort aufgeführten Sonderregelungen gelten uneingeschränkt für den Betrieb zu Einsatzzwecken. Beim Einsatz von unbemannten Fahrzeugen zu Übungs- und Ausbildungszwecken ist die Notwendigkeit zur Inanspruchnahme der Sonderregelungen zu prüfen und gegebenenfalls darauf zu verzichten.

\subsubsection{Wasser- und Bodenfahrzeuge}

Im Bereich der unbemannten Wasserfahrzeuge sind zukünftig alle noch ausstehenden Gesetze und Verordnungen, nach Herausgabe durch den Gesetzgeber, zur Regulierung von Einsätzen von Robotik innerhalb oder auf Gewässern, zu berücksichtigen. Nach Bekanntwerden sind diese zeitnah im vorliegenden Konzept zu integrieren.\\

\noindent Für alle genannten Gesetze und Verordnungen sind insbesondere die Ausnahmen für Behörden regelmäßig zu prüfen und entsprechend anzuwenden.\\

\noindent Selbiges gilt für Bodenfahrzeuge.

\subsubsection{Grundsätze Datenschutz}

Durch die Verwendung bildgebender Verfahren, unabhängig davon ob dies zur Steuerung von unbemannten Systemen oder zur Nutzung im Einsatz erfolgt, ist eine besondere Sensibilität hinsichtlich des Datenschutzes zu beachten.

\paragraph{Rechtmäßigkeit der Verarbeitung}

Im Rahmen der allgemeinen feuerwehrtechnischen Tätigkeit kann nach §\,6 Abs. 1\,(d) DSGVO und §\,6 Abs. 1\,(e) DSGVO\footnote{\url{https://dsgvo-gesetz.de/art-6-dsgvo/}} von der Wahrnehmung einer Aufgabe im öffentlichen Interesse ausgegangen werden.

\paragraph{Gebot zur Datensparsamkeit}

Auf Basis der DSGVO gilt allgemein das Prinzip der Datenminimierung, dass also nur solche Daten erhoben oder gespeichert werden, die zur Erfüllung der Aufgaben notwendig sind. Auf dieser Grundlage sind nach einem Einsatz nicht weiter benötigte Daten zu löschen.\\

\noindent Ausdrücklich ausgenommen sind hiervon Daten, welche im Rahmen der weiteren Ausbildung genutzt werden, solange aus diesen keine personenbezogenen Informationen gewonnen werden können.

\subsubsection{Wichtige Einschränkungen und Ausnahmen für Luftfahrzeuge}

Für den Flugbetrieb in Kontrollzonen (Lufträume der Klasse D-CTR) muss vor dem Betrieb von eine Flugverkehrskontrollfreigabe bei der zuständigen Flugverkehrskontrollstelle eingeholt werden. Hierzu können \ac{nfl} der Flugsicherungsorganisationen herangezogen werden.\\

\noindent Für den Flugbetrieb in Kontrollzonen von Militärflugplätzen oder anderem militärischem Sperrgebiet ist für jeden Start eine individuelle Genehmigung einzuholen.

\subsection{Einsatzgebiet und Gültigkeitsraum}

Die \callee{} steht im gesamten Kreisgebiet \district{} zur Verfügung. Die überregionale Hilfeleistung ist unter Berücksichtigung und Vorgabe von Artikel §35 GG\footnote{\url{https://www.gesetze-im-internet.de/gg/art_35.html}} möglich.
